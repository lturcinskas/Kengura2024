%S1===============================[=.............................]===============================%S1
\def\SoneEN{A pattern is made of equal pentagons. Which of the tiles below, when placed in the central hole, will form a self-intersecting loop?}
\def\SoneLT{Elzė sudarė figūrą iš 8 vienodų detalių ir ant jos neatitraukdama pieštuko nubrėžė uždarą liniją, kuri kerta pati save. Elzė pašalino vidurinę detalę, kaip parodyta paveikslėlyje dešinėje. Kaip atrodo pašalintoji detalė?}
\def\SonePL{}
\def\SoneRU{}

\def\rSone{C}
 \def\aSone{\granswer[0.4]{J02-1}{J02-2}{J02-3}{J02-4}{J02-5}}

%S2===============================[==............................]===============================%S2
\def\StwoEN{Which of these integers is two less than a multiple of ten, two more than a square, and two times a prime?}
\def\StwoLT{Kurį skaičių padidinus 2, gaunamas skaičiaus 10 kartotinis, sumažinus 2 -- sveikojo skaičiaus kvadratas, o sumažinus 2 kartus -- pirminis skaičius?}
\def\StwoPL{}
\def\StwoRU{}

\def\rStwo{C}
\def\aStwo{\answer{$78$}{$58$}{$38$}{$18$}{$6$}}

%S3===============================[===...........................]===============================%S3
\def\SthreeEN{A young kangaroo cut a pizza into six equal slices. After eating one slice, he arranged the remaining slices with equal gaps between slices. What size is the angle of each gap?}
\def\SthreeLT{Luka supjaustė picą į 6 lygius gabalus (tiesiais pjūviais nuo picos centro). Vieną iš jų suvalgiusi, likusius ji išdėstė lygiais tarpais, kaip parodyta paveikslėlyje. Kampas tarp dviejų gretimų picos gabalų lygus}
\def\SthreePL{}
\def\SthreeRU{}

\def\rSthree{E}
\def\aSthree{\answer{$5^\circ$}{$8^\circ$}{$9^\circ$}{$10^\circ$}{$12^\circ$}}

%S4===============================[====..........................]===============================%S4
\def\SfourEN{Kaito has manipulated a die. The probabilities of rolling a 2, 3, 4 or 5 are still $\dfrac{1}{6}$ each, but the probability of rolling a 6 is twice the probability of rolling a 1. What is the probability of rolling a 6?}
\def\SfourLT{Augustas šiek tiek pakeitė standartinio lošimo kauliuko formą. Tikimybės, kad paridentas kauliukas atvirs 2, 3, 4 arba 5 akutėmis, nepakito -- jos lygios $\dfrac{1}{6}$. Kokia dabar yra tikimybė, kad kauliukas atvirs 6 akutėmis, jei ji yra du kartus didesnė nei tikimybė, kad jis atvirs viena akute?}
\def\SfourPL{}
\def\SfourRU{}

\def\rSfour{D}
\def\aSfour{\answer{$\dfrac{1}{4}$}{$\dfrac{1}{6}$}{$\dfrac{7}{36}$}{$\dfrac{2}{9}$}{$\dfrac{5}{18}$}}

%S5===============================[=====.........................]===============================%S5
\def\SfiveEN{Tim wants to draw the figure shown on a piece of paper, without lifting his pencil off the paper.  The lengths of the lines are given in the figure.  He can choose to start his drawing anywhere.  What is the shortest distance he could draw to complete the figure?}
\def\SfiveLT{Neatitraukdamas pieštuko nuo popieriaus lapo, Tomas nubrėžė figūrą, kurią sudaro šešios atkarpos. Paveikslėlyje parodyta ši figūra ir visų atkarpų ilgiai. Kokį trumpiausią kelią galėjo popieriumi nueiti pieštukas, Tomui brėžiant figūrą?}
\def\SfivePL{}
\def\SfiveRU{}

\def\rSfive{B}
\def\aSfive{\answer{$14\ \textup{cm}$}{$15\ \textup{cm}$}{$16\ \textup{cm}$}{$17\ \textup{cm}$}{$18\ \textup{cm}$}}

%S6===============================[======........................]===============================%S6
\def\SsixEN{Which of the expressions below has the same value as $16^{15} + 16^{15} + 16^{15} + 16^{15}$?}
\def\SsixLT{Suma \,$16^{15} + 16^{15} + 16^{15} + 16^{15}$\, yra lygi}
\def\SsixPL{}
\def\SsixRU{}

\def\rSsix{A}
\def\aSsix{\answer{$4^{31}$}{$4^{60}$}{$4^{122}$}{$16^{19}$}{$16^{60}$}}

%S7===============================[=======.......................]===============================%S7
\def\SsevenEN{Juuso has an unusual habit of drawing the $xy$-plane with the positive coordinate axes pointing left and down. What would the graph of the equation $y=x+1$ look like in a coordinate system drawn by Juuso?}
\def\SsevenLT{Jokūbas tyrinėja stačiakampę koordinačių sistemą $Oxy$, kurios ašys nukreiptos priešingomis kryptimis nei įprasta: $Ox$ į kairę, $Oy$ žemyn. Kuriame paveikslėlyje pavaizduota tiesė, nusakoma lygtimi $y=x+1$?}
\def\SsevenPL{}
\def\SsevenRU{}

\def\rSseven{D}
\def\aSseven{\granswertoig[0.7]{S07-1}{S07-2}{S07-3}{S07-4}{S07-5}}

%S8===============================[========......................]===============================%S8
\def\SeightEN{There are $6$ glasses on a table with their open ends up. In any one move, we turn over exactly $4$ of them. What is the least number of moves required to have all glasses upside down?}
\def\SeightLT{Ant stalo dugnu aukštyn stovi $6$ stiklinės. Vienu ėjimu leidžiama pasirinkti bet kurias 4 stiklines ir jas apversti. Per kiek mažiausiai ėjimų galima pasiekti, kad visos 6 stiklinės stovėtų dugnu žemyn?}
\def\SeightPL{}
\def\SeightRU{}

\def\rSeight{B}
\def\aSeight{\answer{$2$}{$3$}{$4$}{$5$}{$6$}}

%S9===============================[=========.....................]===============================%S9
\def\SnineEN{A black trail and a grey trail cross a park, as shown. Each trail divides the park into two regions of equal area.  Which of the following must be true about the areas $A$, $B$ and $C$?}
\def\SnineLT{Per pievą eina du takeliai. Kiekvienas iš jų dalija pievą į dvi lygiaplotes dalis. Trijų pievos sričių, kurias riboja takeliai, plotai lygūs $A$, $B$ ir $C$, kaip parodyta paveikslėlyje (čia takeliai pažymėti skirtingomis linijomis).  Kuri lygybė yra garantuotai teisinga?}
\def\SninePL{}
\def\SnineRU{}

\def\rSnine{B}
\def\aSnine{\answer{$A=C$}{$B=A+C$ }{$B=\dfrac{1}{2}(A+C)$\\}{$B=\dfrac{2}{3}(A+C)$}{$B=\dfrac{3}{5}(A+C)$}}

%S10===============================[==========....................]===============================%S10
\def\StenEN{Beaver wishes to color the squares and triangles of the following figure so that no two neighbouring figures, even those sharing a single vertex, are the same color. \newline
\centerline{PAV}\newline
What is the least number of colors needed?}
\def\StenLT{Ema iš plytelių sudėjo stačiakampį (žr.~pav.). Ji panaudojo kelių spalvų kvadratines ir trikampes plyteles. Kiekviena plytelė yra vienspalvė.  Kiekvienos dvi plytelės, turinčios sąlyčio tašką (net jei vienintelį), yra skirtingų spalvų. Kiek mažiausiai spalvų gali turėti Emos sudėtas stačiakampis?}
\def\StenPL{}
\def\StenRU{}

\def\rSten{C}
\def\aSten{\answer{$3$}{$4$}{$5$}{$6$}{$7$}}

%S11===============================[===========...................]===============================%S11
\def\SuoneEN{A triangular pyramid $ABCD$ has sides of length $5$, $6$, $7$, $8$, $9$ and $10$. The points $M$, $N$, $P$, $Q$, $R$ and $S$ are the midpoints of the edges of the pyramid, as shown. \newline
\centerline{PAV}
\newline
What is the perimeter of the closed hexagonal line $MNPQRSM$?}
\def\SuoneLT{Trikampės piramidės $ABCD$ briaunų vidurio taškai $M$, $N$, $P$, $Q$, $R$ ir $S$ sujungti atkarpomis, kaip parodyta paveikslėlyje. Šios atkarpos sudaro uždarą laužtę $MNPQRSM$. Koks yra jos ilgis, jei $AD=5$, \,$AC=6$, \,$AB=7$, \,$CD=8$, \,$BD=9$ \,ir\, $BC=10$?}
\def\SuonePL{}
\def\SuoneRU{}

\def\rSuone{C}
\def\aSuone{\answer{$19$}{$20$}{$21$}{$22$}{Kitas atsakymas}}

%S12===============================[============..................]===============================%S12
\def\SutwoEN{A student started with the number $1$ and multiplied it by either $6$ or $10$. He then multiplied the result by either $6$ or $10$, and continued this procedure many times. Which of the following cannot be one of the numbers he obtained?}
\def\SutwoLT{Jei lentoje užrašytas skaičius $n$, tai leidžiama jį nutrinti bei užrašyti vieną iš skaičių $6n$ ir~$10n$.
Kurio skaičiaus neįmanoma gauti tokiu būdu, pradžioje turint užrašytą skaičių~1?}
\def\SutwoPL{}
\def\SutwoRU{}

\def\rSutwo{B}
\def\aSutwo{\answer{$2^{100}\cdot3^{20}\cdot5^{80}$  }{$2^{90}\cdot3^{20}\cdot5^{80}$ }{$2^{110}\cdot3^{80}\cdot5^{30}$  }{$2^{90}\cdot3^{20}\cdot5^{70}$ }{$2^{50}\cdot5^{50}$ }}

%S13===============================[=============.................]===============================%S13
\def\SuthreeEN{Sanjay cuts out three circles from three different pieces of coloured card. He places them on top of each other, as shown in Figure 1. He then moves the circles so that all three circles are tangent to each other, as shown in Figure 2. In the first figure, the area of the visible black region is seven times the area of the white circle. What is the ratio between the areas of the visible black regions in the two figures?}
\def\SuthreeLT{Rokas tris popierinius skritulius pradžioje sudėjo, kad sutaptų skritulių centrai, o vėliau -- kad bet kurie du skritulius ribojantys apskritimai liestųsi (žr.~pav.). Pirmosios gautos figūros juodosios srities plotas yra 7 kartus didesnis nei baltosios srities plotas. Koks yra dviejų gautųjų figūrų juodųjų sričių plotų santykis?}
\def\SuthreePL{}
\def\SuthreeRU{}

\def\rSuthree{D}
\def\aSuthree{\answer{$3:1$}{$4:3$}{$6:5$}{$7:6$}{$9:7$}}

%S14===============================[==============................]===============================%S14
\def\SufourEN{Exactly one of these statements about a certain positive integer $n$ is true. Which statement is true?}
\def\SufourLT{Yra žinoma, kad lygiai vienas iš teiginių \textbf{A}--\textbf{E} apie tam tikrą natūralųjį skaičių $n$ yra teisingas. Kuris?}
\def\SufourPL{}
\def\SufourRU{}

\def\rSufour{C}
\def\aSufour{\answer{Skaičius $n$ dalijasi iš 3}{Skaičius $n$ dalijasi iš 6}{Skaičius $n$ yra nelyginis}{$n = 2$}{Skaičius $n$ yra pirminis}}
\def\aSufourEXP{\answer{Skaičius $n$ dalijasi iš 3}{Skaičius $n$ dalijasi iš 6\\}{Skaičius $n$ yra nelyginis}{$n = 2$}{Skaičius $n$ yra pirminis}}

%S15===============================[===============...............]===============================%S15
\def\SufiveEN{John has a number of all black or all white unit cubes and wants to build a $3\times 3\times 3$ cube using $27$ of them. He wants the surface to be exactly half black and half white. What is the smallest number of black cubes he can use?}
\def\SufiveLT{Sofija turi sudėti $3\times 3\times 3$ kubą iš 27 vienodų kubelių. Kiekvieną kubelį ji turi nudažyti viena iš dviejų spalvų: raudonai arba mėlynai. Sudėjus kubą, lygiai pusė jo paviršiaus turi būti raudona ir lygiai pusė -- mėlyna. Kiek mažiausiai kubelių turi būti nudažyta raudonai?}
\def\SufivePL{}
\def\SufiveRU{}

\def\rSufive{E}
\def\aSufive{\answer{$9$}{$11$}{$12$}{$14$}{Kitas atsakymas}}

%S16===============================[================..............]===============================%S16
\def\SusixEN{A diagonal, a semicircle and a quadrant are drawn in a square of side 6 cm. What is the area, in cm$^2$, of the shaded part?}
\def\SusixLT{Kvadrato įstrižainė, pusapskritimis ir apskritimo ketvirtis dalija kvadratą į 6 dalis (žr.~pav.). Koks yra užtušuotos srities plotas, jei kvadrato kraštinės ilgis lygus 6?}
\def\SusixPL{}
\def\SusixRU{}

\def\rSusix{A}
\def\aSusix{\answer{$9$}{$3\pi$}{$6\pi-9$}{$\dfrac{10\pi}{3}$}{$12$}}

%S17===============================[=================.............]===============================%S17
\def\SusevenEN{We have two positive numbers $p$ and $q$, with $p<q$. Which of these expressions is the largest?}
\def\SusevenLT{Kurios trupmenos reikšmė didžiausia, jei skaičiams $p$ ir $q$ galioja nelygybės $0<p<q$?}
\def\SusevenEXP{Kurios trupmenos reikšmė didžiausia, jei skaičiams $p$ ir $q$ galioja nelygybės ${0<p<q}$?}
\def\SusevenPL{}
\def\SusevenRU{}

\def\rSuseven{A}
\def\aSuseven{\answer{$\dfrac{p+3q}4$}{$\dfrac{p+2q}3$}{$\dfrac{p+q}2$}{$\dfrac{2p+q}3$}{$\dfrac{3p+q}4$}}
\def\aSusevenEXP{\fillanswer{$\dfrac{p+3q}4$}{$\dfrac{p+2q}3$}{$\dfrac{p+q}2$}{$\dfrac{2p+q}3$}{$\dfrac{3p+q}4$}}

%S18===============================[==================............]===============================%S18
\def\SueightEN{I write down a 4-digit non-zero number $N=\overline{pqrs}$. When I place a decimal point between the $q$ and the $r$, I find that the resulting number $\overline{pq.rs}$ is the average of the two-digit numbers $\overline{pq}$ and $\overline{rs}$. What is the sum of the digits of $N$?}
\def\SueightLT{Dviženklių skaičių $\overline{AB}$ ir $\overline{CD}$ aritmetinis vidurkis gaunamas, keturženklį skaičių \mbox{$N=\overline{ABCD}$} padalijus iš 100. Kokia yra skaičiaus $N$ skaitmenų suma?}
\def\SueightPL{}
\def\SueightRU{}

\def\rSueight{B}
\def\aSueight{\answer{$14$}{$18$}{$21$}{$25$}{$27$}}

%S19===============================[===================...........]===============================%S19
\def\SunineEN{Kangaroo solves the equation $ax^2 + bx + c = 0$, and Beaver solves the equation $bx^2 + ax + c = 0$, where $a, b, c$ are pairwise distinct non-zero integers. It turns out that the equations share a solution. Which of the following must be true?}
\def\SunineLT{Duoti trys sveikieji skaičiai $a$, $b$, $c$, nelygūs 0. Jokie du iš jų nėra lygūs.
Lygtys ${ax^2 + bx + c = 0}$ \,ir\, $bx^2 + ax + c = 0$ turi bendrą sprendinį $x=x_0$.  Kuris teiginys yra garantuotai teisingas?}
\def\SuninePL{}
\def\SunineRU{}

\def\rSunine{E}
\def\aSunine{\answer{$x_0=0$}{Lygtis $ax^2 + bx + c = 0$ turi vienintelį sprendinį}{$a>0$}{$b<0$}{$a+b+c=0$}}

%S20===============================[====================..........]===============================%S20
\def\SutenEN{There are four vases on the table in which a number of sweets have been placed.
\newline
The number of sweets in the first vase is the number of vases that contain one sweet. \newline
The number of sweets in the second vase is equal to the number of vases that contain two sweets. \newline
The number of sweets in the third vase is equal to the number of vases that contain three sweets.  \newline
The number of sweets in the fourth vase is equal to the number of vases that contain zero sweets. 
\newline
How many sweets are in all the vases together?}
\def\SutenLT{Nojus laiko saldainius keturiose striukės kišenėse. Jis užrašė, po kiek saldainių yra kiekvienoje kišenėje. Jo sesuo Lėja užrašė, keliose kišenėse yra lygiai vienas saldainis, keliose lygiai du, keliose lygiai trys, o keliose -- nė vieno saldainio. Lėja užrašė tuos pačius keturis skaičius kaip ir Nojus. Kiek iš viso saldainių yra striukės kišenėse?}
\def\SutenPL{}
\def\SutenRU{}

\def\rSuten{C}
\def\aSuten{\answer{2}{3}{4}{5}{6}}

%S21===============================[=====================.........]===============================%S21
\def\SdoneEN{ How many three-digit numbers are there that contain at least one of the digits $1$, $2$ or $3$?}
\def\SdoneLT{Kiek yra natūraliųjų triženklių skaičių, turinčių bent vieną iš skaitmenų $1$, $2$ ir $3$?}
\def\SdonePL{}
\def\SdoneRU{}

\def\rSdone{E}
\def\aSdone{\answer{$27$}{$147$}{$441$}{$557$}{$606$}}

%S22===============================[======================........]===============================%S22
\def\SdtwoEN{The figure shows four squares. The smaller ones have side lengths $a$, $b$ and $c$. The vertices $A$ and $C$ of two of the smaller squares coincide with two diagonally opposite vertices of the large square. The vertex $B$ of the third small square is on the side of the large one. Which of the following expressions represents the side length of the largest square?}
\def\SdtwoLT{Tarp trijų mažesniųjų kvadratų, kurių kraštinių ilgiai yra $a$, $b$ ir $c$, įbrėžtas didysis kvadratas, kaip parodyta paveikslėlyje. Didžiojo kvadrato kraštinės ilgis lygus}
\def\SdtwoPL{}
\def\SdtwoRU{}

\def\rSdtwo{D}
\def\aSdtwo{\answer{$\dfrac{2}{3}(a+b+c)$}{$\sqrt{a^2+b^2+c^2}$}{$\sqrt {(b-a)^2+c^2}$}{\ $\sqrt {(a+b)^2+c^2}$\hspace{4pt}}{ $\sqrt {a^2+ab + b^2+c^2}$}}

%S23===============================[=======================.......]===============================%S23
\def\SdthreeEN{Rasika has several unbiased 12-sided dice,
 each with faces labelled $1$ to $12$. When rolling all the dice at once,  the probability of rolling a $12$ exactly once is equal to the probability of rolling no $12$s. How many dice does Rasika have?}
\def\SdthreeLT{Augustė turi kelis vienodus dvylikasienius lošimo kauliukus. Paridenus tokį kauliuką, jis atvirsta vienu iš skaičių $1$, 2, $\ldots$, $12$, ir visi šie skaičiai yra vienodai tikėtini. Paridenus visus Augustės kauliukus, yra vienodai tikėtina, kad skaičiumi 12 atvirs lygiai vienas kauliukas ir kad šiuo skaičiumi neatvirs nė vienas kauliukas. Kiek kauliukų turi Augustė?}
\def\SdthreePL{}
\def\SdthreeRU{}

\def\rSdthree{D}
\def\aSdthree{\answer{$8$}{$9$}{$10$}{$11$}{$12$}}

%S24===============================[========================......]===============================%S24
\def\SdfourEN{Andre has six cards with one number written on each side of each card. The pairs of numbers on the cards are  $(5,12), (3,11), (0,16), (7,8), (4,14)$ and $(9,10)$. The cards can be placed in any order in the blank spaces of the figure.
\centerline{PAV}
What is the smallest result he can get?}
\def\SdfourLT{Domas turi šešias korteles.
Kiekvienos kortelės abiejose pusėse yra po skaičių. Skaičių poros, esančios kortelėse, yra tokios: $(5,12)$, $(3,11)$, $(0,16)$, $(7,8)$, $(4,14)$ ir $(9,10)$. 
Domas tam tikra tvarka sudėjo korteles langeliuose (žr.~pav.), atvertęs kiekvieną iš jų vienu iš dviejų atitinkamų skaičių, ir apskaičiavo gautojo reiškinio reikšmę.
\vspace{-3pt}
\begin{center}
 \includegraphics[scale=0.6]{S24-1}
 \vspace{-5pt}
\end{center}
Kokią mažiausią reikšmę galėjo gauti Domas?}
\def\SdfourEXP{Domas turi šešias korteles.
Kiekvienos kortelės abiejose pusėse yra po skaičių. Skaičių poros, esančios kortelėse, yra tokios: $(5,12)$, $(3,11)$, $(0,16)$, $(7,8)$, $(4,14)$ ir $(9,10)$.
Domas tam tikra tvarka sudėjo korteles langeliuose (žr.~pav.), atvertęs kiekvieną iš jų vienu iš dviejų atitinkamų skaičių, ir apskaičiavo gautojo reiškinio reikšmę.
\vspace{-8pt}
\begin{center}
 \includegraphics[scale=0.8]{S24-1}
 \vspace{-5pt}
\end{center}
Kokią mažiausią reikšmę galėjo gauti Domas?}
\def\SdfourPL{}
\def\SdfourRU{}

\def\rSdfour{D}
\def\aSdfour{\answer{$-23$}{$-24$}{$-25$}{$-26$}{$-27$}}

%S25===============================[=========================.....]===============================%S25
\def\SdfiveEN{Two candles of equal length start burning at the same time. One of the candles will burn down in $4$ hours, the other in $5$ hours, each at their own constant rate. How many hours will they have to burn before one candle is $3$ times the length of the other?}
\def\SdfiveLT{Dvi to paties ilgio žvakės uždegtos vienu metu. Jos dega skirtingais pastoviais greičiais: viena sudegs per $4$ val., kita -- per $5$ val. Kiek laiko degs žvakės, kol viena taps 3 kartus ilgesnė už kitą?}
\def\SdfivePL{}
\def\SdfiveRU{}

\def\rSdfive{A}
\def\aSdfive{\answer{$\dfrac{40}{11}$~val.}{$\dfrac{45}{12}$~val.}{$\dfrac{63}{20}$~val.}{$3$~val.}{$\dfrac{47}{14}$~val.}}

%S26===============================[==========================....]===============================%S26
\def\SdsixEN{A polynomial $p(x)$ satisfies the relation $p(x+1)= x^2-x+2p(6)$ for every real $x$. What is the sum of the coefficients of $p$?}
\def\SdsixLT{Daugianariui $p(x)$ lygybė $p(x+1)= x^2-x+2p(6)$ galioja su kiekvienu realiuoju $x$. Kokia yra daugianario $p(x)$ visų koeficientų suma?}
\def\SdsixPL{}
\def\SdsixRU{}

\def\rSdsix{C}
\def\aSdsix{\answer{$-6$}{$12$}{$-40$}{$40$}{Kitas atsakymas}}

%S27===============================[===========================...]===============================%S27
\def\SdsevenEN{The values of $x,y$ and $z$ satisfy $2^x=3$, $2^y=7$ and $6^z=7$.  Which of the following gives the relationship between $x,y$ and $z$?}
\def\SdsevenLT{Kuri lygybė sieja skaičius $x$, $y$, $z$, jei $2^x=3$, \,$2^y=7$ \,ir\, $6^z=7$?}
\def\SdsevenPL{}
\def\SdsevenRU{}

\def\rSdseven{A}
\def\aSdseven{\answer{$z=\dfrac y{1+x}$}{$z=\dfrac x{y}+1$}{$z=\dfrac yx-1$}{$z=\dfrac x{y-1}$}{$z=y-\dfrac1x$}}

%S28===============================[============================..]===============================%S28
\def\SdeightEN{A function $f \colon \mathbb R \longrightarrow \mathbb R$ satisfies $f(20-x)=f(22+x)$ for all real $x$. It is known that $f$ has exactly two roots. What is the sum of these two roots?}
\def\SdeightLT{Funkcijai $f \colon \mathbb R \rightarrow \mathbb R$ lygybė $f(20-x)=f(22+x)$ galioja su kiekvienu realiuoju $x$. Egzistuoja lygiai dvi $x$ reikšmės, kurioms $f(x)=0$. Kokia yra šių dviejų reikšmių suma?}
\def\SdeightPL{}
\def\SdeightRU{}

\def\rSdeight{E}
\def\aSdeight{\answer{$-1$}{$20$}{$21$}{$22$}{$42$}}

%S29===============================[=============================.]===============================%S29
\def\SdnineEN{Twelve points are equally spaced on a circle.  How many triangles containing a $45^\circ$ angle can be formed by choosing three of these points?}
\def\SdnineLT{Liepa pažymėjo 12 apskritimo taškų, kad atstumai tarp gretimų taškų būtų lygūs. Ji nori sujungti tris iš pažymėtųjų taškų, kad gautasis trikampis turėtų $45^\circ$ kampą. Kiek tokių trikampių ji gali gauti?}
\def\SdninePL{}
\def\SdnineRU{}

\def\rSdnine{D}
\def\aSdnine{\answer{$48$}{$60$}{$72$}{$84$}{Kitas atsakymas}}

%S30===============================[==============================]===============================%S30
\def\SdtenEN{A special four-digit number $\overline{abcd}$ satisfies the equation $\overline{abcd} = a^{a}+b^b+c^c+d^d$.
What is the value of $a$?}
\def\SdtenLT{Keturženklis skaičius $\overline{ABCD}$, neturintis skaitmens 0, lygus $A^{A}+B^B+C^C+D^D$.
Tada~$A=$}
\def\SdtenPL{}
\def\SdtenRU{}

\def\rSdten{B}
\def\aSdten{\answer{$2$}{$3$}{$4$}{$5$}{$6$}}

