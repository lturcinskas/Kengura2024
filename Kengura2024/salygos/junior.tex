%J1===============================[=.............................]===============================%J1
\def\JoneEN{What is the value of $\dfrac{2 \times 0.24}{20 \times 2.4}$?}
\def\JoneLT{Kokia yra reiškinio \,$\dfrac{2 \cdot 0{,}24}{20 \cdot 2{,}4}$\, reikšmė?}
\def\JonePL{}
\def\JoneRU{}

\def\rJone{A}
\def\aJone{\answer{$0{,}01$}{$0{,}1$}{$1$}{$10$}{$100$}}

%J2===============================[==............................]===============================%J2
\def\JtwoEN{A pattern is made of equal pentagons. Which of the tiles below, when placed in the central hole, will form a self-intersecting loop?}
\def\JtwoLT{Elzė sudarė figūrą iš 8 vienodų detalių ir ant jos neatitraukdama pieštuko nubrėžė uždarą liniją, kuri kerta pati save. Elzė pašalino vidurinę detalę, kaip parodyta paveikslėlyje dešinėje. Kaip atrodo pašalintoji detalė?}
\def\JtwoPL{}
\def\JtwoRU{}

\def\rJtwo{C}
\def\aJtwo{\granswer[0.4]{J02-1}{J02-2}{J02-3}{J02-4}{J02-5}}

%J3===============================[===...........................]===============================%J3
\def\JthreeEN{The number of the dots on opposite faces of a die add to $7$. The vertex labelled $P$ on the die is formed by the faces which have $1$, $2$ and $3$ dots on them. Its vertex sum is the sum of the number of dots on those faces which meet at a given vertex. The vertex sum of $P$ is $1+2+3=6$.\newline
What is the maximum of the vertex sums of vertices $Q$, $R$ and $S$?}
\def\JthreeLT{Paveikslėlyje pavaizduotas lošimo kauliukas, kurio kiekvienose dviejose priešingose sienelėse yra lygiai 7 akutės. Pasirinkus tris sieneles, turinčias bendrą viršūnę, jai priskiriamas tų trijų sienelių bendras akučių skaičius. Pavyzdžiui, viršūnės $P$ skaičius yra $1+2+3=6$. Linas apskaičiavo viršūnių $Q$, $R$ ir $S$ skaičius. Didžiausias iš jų yra}
\def\JthreePL{}
\def\JthreeRU{}

\def\rJthree{D}
\def\aJthree{\answer{ $7$}{$9$}{$10$}{$11$}{$15$}}

%J4===============================[====..........................]===============================%J4
\def\JfourEN{A hopping game is played in the following way: Each player hops into the squares, swapping between left foot - both feet - right foot - both feet - left foot - both feet, and so on, as shown. Maya played the game and hopped into exactly $48$ squares starting with her left foot. How many times did her left foot touch the ground?}
\def\JfourLT{Milda ant žemės viena eile nubrėžė 48 langelius ir šokinėja iš vieno langelio į kitą. Į~pirmąjį langelį ji įšoka kaire koja, į antrąjį -- abiem, į trečiąjį -- dešine, į~ketvirtąjį -- abiem, o toliau šiuos šuolius vis kartoja: į penktąjį langelį vėl įšoka kaire koja, į šeštąjį -- vėl abiem, ir t.~t. (žr. pav.). Keliuose langeliuose Milda paliečia žemę kaire koja?}
\def\JfourPL{}
\def\JfourRU{}

\def\rJfour{D}
\def\aJfour{\answer{12}{24}{32}{36}{40}}

%J5===============================[=====.........................]===============================%J5
\def\JfiveEN{Tim wants to draw the figure shown on a piece of paper, without lifting his pencil off the paper.  The lengths of the lines are given in the figure.  He can choose to start his drawing anywhere.  What is the shortest distance he could draw to complete the figure?}
\def\JfiveLT{Neatitraukdamas pieštuko nuo popieriaus lapo, Tomas nubrėžė figūrą, kurią sudaro šešios atkarpos. Paveikslėlyje parodyta ši figūra ir visų atkarpų ilgiai. Kokį trumpiausią kelią galėjo popieriumi nueiti pieštukas, Tomui brėžiant figūrą?}
\def\JfivePL{}
\def\JfiveRU{}

\def\rJfive{B}
%\def\aJfive{\answer{{}{}{}{}{}}}
\def\aJfive{\answer{$14\ \textup{cm}$}{$15\ \textup{cm}$}{$16\ \textup{cm}$}{$17\ \textup{cm}$}{$18\ \textup{cm}$}}

%J6===============================[======........................]===============================%J6
\def\JsixEN{John makes a sequence of structures on a table, beginning with one cube. He makes the next structure by adding five cubes which hide the visible faces of the initial cube, as shown. What is the smallest number of cubes he needs to add to the second structure so that all the visible faces of the second structure are hidden?}
\def\JsixLT{Ignas ant stalo padėjo kubą, o tada apdėjo jį dar penkiais kubais, pilnai uždengdamas visas penkias matomas pradinio kubo sienas (žr.~pav.). Kiek mažiausiai kubų Ignui prireiks, kad jais apdėtų gautąją figūrą ir pilnai uždengtų visą jos matomą paviršių?}
\def\JsixPL{}
\def\JsixRU{}

\def\rJsix{C}
\def\aJsix{\answer{9}{11}{13}{17}{21}}

%J7===============================[=======.......................]===============================%J7
\def\JsevenEN{The figure shows a square with four circles of equal area, each touching two sides of the square and two other circles.  What is the ratio between the areas of the black region and the grey region?}
\def\JsevenLT{Keturi apskritimai liečia kvadrato kraštines ir vienas kitą, kaip parodyta paveikslėlyje. Koks yra kvadrato juodosios srities ploto ir pilkosios srities ploto santykis?}
\def\JsevenPL{}
\def\JsevenRU{}

\def\rJseven{E}
\def\aJseven{\answer{$1:4$}{$3:4$}{$\pi:1$}{$2:3$}{$1:3$}}

%J8===============================[========......................]===============================%J8
\def\JeightEN{A three-digit palindrome is a number of the form $\text{'}aba\text{'}$ where the digits $a$ and $b$ can either be the same or different. What is the sum of the digits of the largest three-digit palindrome that is also a multiple of 6?}
\def\JeightLT{Natūralusis triženklis skaičius prasideda ir baigiasi tuo pačiu skaitmeniu bei dalijasi iš~6. Didžiausio tokio skaičiaus skaitmenų suma yra}
\def\JeightPL{}
\def\JeightRU{}

\def\rJeight{E}
\def\aJeight{\answer{16}{18}{20}{21}{24}}

%J9===============================[=========.....................]===============================%J9
\def\JnineEN{There are $6$ glasses on a table with their open ends up. In any one move, we turn over exactly $4$ of them. What is the least number of moves required to have all glasses upside down?}
\def\JnineLT{Ant stalo dugnu aukštyn stovi $6$ stiklinės. Vienu ėjimu leidžiama pasirinkti bet kurias 4 stiklines ir jas apversti. Per kiek mažiausiai ėjimų galima pasiekti, kad visos 6 stiklinės stovėtų dugnu žemyn?}
\def\JninePL{}
\def\JnineRU{}

\def\rJnine{B}
\def\aJnine{\answer{$2$}{$3$}{$4$}{$5$}{$6$}}

%J10===============================[==========....................]===============================%J10
\def\JtenEN{Ardal encloses a rectangular field with $40\ \textup{m}$ of fence. The side-lengths of the field are all prime numbers. What is the maximum possible area of the field?}
\def\JtenLT{Sklypas, aptvertas $40\ \textup{m}$ ilgio tvora, yra stačiakampio formos. Šio stačiakampio kraštinių ilgiai metrais yra pirminiai skaičiai. Koks yra didžiausias galimas sklypo plotas?}
\def\JtenPL{}
\def\JtenRU{}

\def\rJten{D}
\def\aJten{\answer{$51\ \textup{m}^2$}{$65\ \textup{m}^2$}{$75\ \textup{m}^2$}{$91\ \textup{m}^2$}{$99\ \textup{m}^2$}}

%J11===============================[===========...................]===============================%J11
\def\JuoneEN{Martin draws a square with vertices $A$, $B$, $C$, $D$ and a regular hexagon with side $OC$, where $O$ is the center of the square. What is the size of angle $\alpha$?}
\def\JuoneLT{Kvadratas $ABCD$ su centru $O$ ir taisyklingasis šešiakampis, turintis kraštinę $OC$, kertasi, kaip parodyta paveikslėlyje. Tada $\alpha=$}
\def\JuonePL{}
\def\JuoneRU{}

\def\rJuone{A}
\def\aJuone{\answer{$105^\circ$}{$110^\circ$}{$115^\circ$}{$120^\circ$}{$125^\circ$}}

%J12===============================[============..................]===============================%J12
\def\JutwoEN{A student started with the number $1$ and multiplied it by either $6$ or $10$. He then multiplied the result by either $6$ or $10$, and continued this procedure many times. Which of the following cannot be one of the numbers he obtained?}
\def\JutwoLT{Jei lentoje užrašytas skaičius $n$, tai leidžiama jį nutrinti bei užrašyti vieną iš skaičių $6n$ ir~$10n$.
Kurio skaičiaus neįmanoma gauti tokiu būdu, pradžioje turint užrašytą skaičių~1?}
\def\JutwoPL{}
\def\JutwoRU{}

\def\rJutwo{B}
\def\aJutwo{\answer{$2^{100}\cdot3^{20}\cdot5^{80}$  }{$2^{90}\cdot3^{20}\cdot5^{80}$ }{$2^{110}\cdot3^{80}\cdot5^{30}$  }{$2^{90}\cdot3^{20}\cdot5^{70}$ }{$2^{50}\cdot5^{50}$ }}

%J13===============================[=============.................]===============================%J13
\def\JuthreeEN{Sanjay cuts out three circles from three different pieces of coloured card. He places them on top of each other, as shown in Figure 1. He then moves the circles so that all three circles are tangent to each other, as shown in Figure 2. In the first figure, the area of the visible black region is seven times the area of the white circle. What is the ratio between the areas of the visible black regions in the two figures?}
\def\JuthreeLT{Rokas tris popierinius skritulius pradžioje sudėjo, kad sutaptų skritulių centrai, o vėliau -- kad bet kurie du skritulius ribojantys apskritimai liestųsi (žr.~pav.). Pirmosios gautos figūros juodosios srities plotas yra 7 kartus didesnis nei baltosios srities plotas. Koks yra dviejų gautųjų figūrų juodųjų sričių plotų santykis?}
\def\JuthreePL{}
\def\JuthreeRU{}

\def\rJuthree{D}
\def\aJuthree{\answer{$3:1$}{$4:3$}{$6:5$}{$7:6$}{$9:7$}}

%J14===============================[==============................]===============================%J14
\def\JufourEN{Jelena places the capital letters A, B, C and D into the $2\times 4$ table shown on the right. Exactly one letter is placed in each cell.
She wishes to make sure that in each row and in each $2\times 2$ square, each of the four letters appears exactly once. In how many ways can she do this?}
\def\JufourLT{Elena turi kiekviename $2\times 4$ lentelės langelyje įrašyti po vieną raidę (žr.~pav.). Kiekvienoje iš dviejų eilučių ir kiekviename iš trijų $2\times 2$ kvadratų turi būti po keturias skirtingas raides A, B, C ir D. Keliais būdais Elena gali užpildyti lentelę?}
\def\JufourPL{}
\def\JufourRU{}

\def\rJufour{B}
\def\aJufour{\answer{12}{24}{48}{96}{198}}

%J15===============================[===============...............]===============================%J15
\def\JufiveEN{A point $P$ is chosen inside an equilateral triangle. From $P$ we draw three segments parallel to the sides, as shown. The lengths of the segments are $2\ \textup{m}$, $3\ \textup{m}$ and $6\ \textup{m}$. What is the perimeter of the triangle?}
\def\JufiveLT{Lygiakraščio trikampio $ABC$ viduje pažymėtas taškas $P$, o iš jo į trikampio kraštines išvestos trys atkarpos. Kiekviena iš trijų atkarpų yra lygiagreti su viena iš trikampio kraštinių, kaip parodyta paveikslėlyje. Jame nurodyti ir šių atkarpų ilgiai. Koks yra trikampio $ABC$ perimetras?}
\def\JufivePL{}
\def\JufiveRU{}

\def\rJufive{C}
\def\aJufive{\answer{$22$}{$26$}{$33$ }{$39$ }{$44$}}

%J16===============================[================..............]===============================%J16
\def\JusixEN{Mary's daughter gave birth to a baby girl today. In two years' time, the product of the ages of Mary, her daughter and her granddaughter will be 2024. Mary's and her daughter's ages are both even numbers. What is Mary's age now?}
\def\JusixLT{Janinos duktė šiandien pagimdė mergaitę.  Janinos ir jos dukters amžiai (metais) yra lyginiai skaičiai. Po dvejų metų Janinos, jos dukters ir anūkės amžių (metais) sandauga bus lygi 2024. Kiek metų šiandien yra Janinai?}
\def\JusixPL{}
\def\JusixRU{}

\def\rJusix{B}
\def\aJusix{\answer{42}{44}{46}{48}{50}}

%J17===============================[=================.............]===============================%J17
\def\JusevenEN{A rectangle is divided into three regions of equal area. One of the regions is an equilateral triangle with side-length $4\ \textup{cm}$, the other two are trapezia, as shown in the figure. What is the length of the smaller of the parallel sides of the trapezia?}
\def\JusevenLT{Stačiakampį sudaro trys lygiaplotės dalys: dvi lygios trapecijos ir lygiakraštis trikampis, kurio kraštinės ilgis yra 4 (žr.~pav.). Koks yra klaustuku pažymėtos atkarpos ilgis?}
\def\JusevenPL{}
\def\JusevenRU{}

\def\rJuseven{D}
\def\aJuseven{\answer{$2$}{$\sqrt2$}{$6-2\sqrt3$}{$\sqrt3$}{Kitas atsakymas}}

%J18===============================[==================............]===============================%J18
\def\JueightEN{A number is written in each of the twelve circles shown. The number inside each square indicates the product of the numbers at its four vertices. What is the product of the numbers in the eight grey circles?}
\def\JueightLT{Kiekviename iš 12 pavaizduotųjų skritulių įrašyta po natūralųjį skaičių. Kiekvienas skaičius, įrašytas kvadrato viduje, lygus keturių skaičių to kvadrato viršūnėse sandaugai. Kokia yra 8 skaičių pilkuosiuose skrituliuose sandauga?}
\def\JueightPL{}
\def\JueightRU{}

\def\rJueight{B}
\def\aJueight{\answer{20}{40}{80}{120\\}{Sandauga gali įgyti daugiau nei vieną reikšmę}}

%J19===============================[===================...........]===============================%J19
\def\JunineEN{Cristina has a set of cards numbered 1 to 12. She places eight of them at the vertices of an octagon so that the sum of every pair of numbers at opposite ends of an edge of the octagon is a multiple of 3. Which numbers did Cristina not place?}
\def\JunineLT{Kęstutis pasirinko 8 iš 12 skaičių 1, 2, $\ldots$, 12 ir juos tam tikra tvarka surašė ratu. Kiekvienų dviejų gretimų to rato skaičių suma dalijasi iš 3. Kurių keturių iš 12 skaičių Kęstutis nepasirinko?}
\def\JuninePL{}
\def\JunineRU{}

\def\rJunine{E}
\def\aJunine{\answer{1, 5, 9, 12 }{3, 5, 7, 9   }{1, 2, 11, 12}{  5, 6, 7, 8}{3, 6, 9, 12  }}

%J20===============================[====================..........]===============================%J20
\def\JutenEN{There are four vases on the table in which a number of sweets have been placed.
\newline
The number of sweets in the first vase is the number of vases that contain one sweet. \newline
The number of sweets in the second vase is equal to the number of vases that contain two sweets. \newline
The number of sweets in the third vase is equal to the number of vases that contain three sweets.  \newline
The number of sweets in the fourth vase is equal to the number of vases that contain zero sweets. 
\newline
How many sweets are in all the vases together?}
\def\JutenLT{Nojus laiko saldainius keturiose striukės kišenėse. Jis užrašė, po kiek saldainių yra kiekvienoje kišenėje. Jo sesuo Lėja užrašė, keliose kišenėse yra lygiai vienas saldainis, keliose lygiai du, keliose lygiai trys, o keliose -- nė vieno saldainio. Lėja užrašė tuos pačius keturis skaičius kaip ir Nojus. Kiek iš viso saldainių yra striukės kišenėse?}
\def\JutenPL{}
\def\JutenRU{}

\def\rJuten{C}
\def\aJuten{\answer{2}{3}{4}{5}{6}}

%J21===============================[=====================.........]===============================%J21
\def\JdoneEN{Jean-Philippe has $n^3 (n>2)$ identical small cubes. He used these to make a large cube and painted the entire outer surface of the large cube. The number of small cubes with only one face painted is equal to the number of those with no face painted. What is the value of $n$?}
\def\JdoneLT{Ernestas nudažė medinį kubą žaliai ir supjaustė jį į $n^3$ vienodų kubelių (čia $n>2$). Taip jis gavo po tiek pat kubelių, turinčių lygiai vieną žalią sienelę, ir kubelių, neturinčių nė vienos žalios sienelės. Kokia yra skaičiaus $n$ reikšmė?}
\def\JdonePL{}
\def\JdoneRU{}

\def\rJdone{D}
\def\aJdone{\answer{4}{6}{7}{8}{Kitas atsakymas}}

%J22===============================[======================........]===============================%J22
\def\JdtwoEN{Carl always tells the truth or always lies on alternate days. One day, he made exactly four of the following five statements. Which one could he not have made on that day?}
\def\JdtwoLT{Mikė Melagėlis nusprendė visą laiką meluoti tik kas antrą dieną, o likusiomis dienomis sakyti tik tiesą. Vieną dieną Mikė pasakė lygiai keturis iš penkių teiginių \textbf{A}--\textbf{E}. Kurio teiginio jis nepasakė?}
\def\JdtwoPL{}
\def\JdtwoRU{}

\def\rJdtwo{A}
\def\aJdtwo{\answer{,,Skaičius 2024 dalijasi iš 11.``\hspace{23pt}}{,,Ir vakar melavau, ir rytoj meluosiu.``\\}
{,,Šiandien ir rytoj kalbu tik tiesą.``}{,,Vakar buvo trečiadienis.``\\}{,,Rytoj bus šeštadienis.``}}

%J23===============================[=======================.......]===============================%J23
\def\JdthreeEN{The prime factorisation of the number $n! = 1 \cdot 2 \cdot \ldots \cdot n$ is of the form shown in the diagram.\newline
\centerline{PAV}\newline
The primes are written in increasing order. Ink has covered some of the primes and some of the exponents. What is the exponent of 17?}
\def\JdthreeLT{Skaičius $n$ yra natūralusis. Skaičius $n! = 1 \cdot 2 \cdot \ldots \cdot n$ yra užrašytas kaip pirminių skaičių sandauga, dauginamuosius rašant didėjimo tvarka:
\vspace{-8pt}
$$2\cdot2\cdot\ldots\cdot11\cdot13\cdot13\cdot13\cdot13\cdot17\cdot\ldots\cdot43\cdot47.\vspace{-8pt}$$
Kiek šioje sandaugoje yra dauginamųjų, lygių 17?}
\def\JdthreePL{}
\def\JdthreeRU{}

\def\rJdthree{C}
\def\aJdthree{\answer{1}{2}{3}{4}{5}}

%J24===============================[========================......]===============================%J24
\def\JdfourEN{Otis makes a net using a combination of squares and equilateral triangles, as show in the figure. The side-length of each square and of each triangle is $1\,\textup{cm}$. He folds the net up into the 3D shape shown. What is the distance between the vertices $A$ and $B$?}
\def\JdfourLT{Įterpus kubą tarp dviejų taisyklingųjų keturkampių piramidžių, gautas erdvinis kūnas, turintis 12 sienų (žr.~pav.). Jo 8 sienos yra lygiakraščiai trikampiai, o kiekvienos briaunos ilgis lygus 1. Koks yra atstumas tarp šio kūno viršūnių $A$ ir $B$?}
\def\JdfourPL{}
\def\JdfourRU{}

\def\rJdfour{E}
\def\aJdfour{\answer{$1+\sqrt{3}$}{$2\sqrt{2}$}{$\dfrac{5}{2}$}{$\sqrt{5}$}{$1+\sqrt{2}$}}

%J25===============================[=========================.....]===============================%J25
\def\JdfiveEN{The sum of the digits of the number $N$ is three times the sum of the digits of the number $N + 1$. What is the smallest possible sum of the digits of $N$?}
\def\JdfiveLT{Natūralųjį skaičių $N$ padidinus 1, jo skaitmenų suma sumažėjo 3 kartus. Kokia yra mažiausia galima skaičiaus $N$ skaitmenų suma?}
\def\JdfivePL{}
\def\JdfiveRU{}

\def\rJdfive{B}
\def\aJdfive{\answer{9}{12}{15}{18}{27}}


%J26===============================[==========================....]===============================%J26
\def\JdsixEN{Jill has some black, gray, and white unit cubes. She uses $27$ of them to build a $3\times 3\times 3$ cube. She wants the surface to be exactly one-third black, one-third gray, and one-third white.  The smallest possible number of black cubes she can use is $A$ and the largest possible number of black cubes she can use is $B$.  What is the value of $B - A$?}
\def\JdsixLT{Sofija turi sudėti $3\times 3\times 3$ kubą iš 27 vienodų kubelių. Kiekvieną kubelį ji turi nudažyti viena iš trijų spalvų: geltonai, žaliai arba raudonai. Sudėjus kubą, lygiai trečdalis jo paviršiaus turi būti geltonas, lygiai trečdalis -- žalias ir lygiai trečdalis -- raudonas. Sofija nustatė, kad geltonai jai reikia nudažyti mažiausiai $m$ kubelių, o daugiausiai $M$ kubelių. Kam lygus skirtumas $M-m$?}
\def\JdsixPL{}
\def\JdsixRU{}

\def\rJdsix{E}
\def\aJdsix{\answer{1}{3}{6}{9}{Kitas atsakymas}}


%J27===============================[===========================...]===============================%J27
\def\JdsevenEN{Twenty points are equally spaced on the circumference of a circle. 
David draws all the possible chords that connect pairs of these points.  How many of these chords are longer than the radius of the circle but shorter than its diameter?}
\def\JdsevenLT{Lukas pažymėjo 20 apskritimo taškų, kad atstumai tarp gretimų taškų būtų lygūs. Kiekvieną pažymėtųjų taškų porą jis sujungė atkarpa. Kiek atkarpų, ilgesnių nei apskritimo spindulys, bet trumpesnių nei jo skersmuo, nubrėžė Lukas?}
\def\JdsevenPL{}
\def\JdsevenRU{}

\def\rJdseven{C}
\def\aJdseven{\answer{90}{100}{120}{140}{160}}

%J28===============================[============================..]===============================%J28
\def\JdeightEN{Olya walked in the park. She walked half of the total time at a speed of 2 km/h. She walked half of the total distance at a speed of 3 km/h. She walked the rest of the time at a speed of 4 km/h. For what fraction of the total time did she walk at a speed of 4 km/h?}
\def\JdeightLT{Ona pasivaikščiojo parke. Pusę parke praleisto laiko ji ėjo 2 km/h greičiu. Pusę parke nueito atstumo ji ėjo 3 km/h greičiu. Likusį laiką Ona ėjo 4 km/h greičiu. Kurią pasivaikščiojimo laiko dalį Ona ėjo 4 km/h greičiu?}
\def\JdeightPL{}
\def\JdeightRU{}

\def\rJdeight{D}
\def\aJdeight{\answer{$\dfrac{1}{4}$}{$\dfrac{1}{5}$}{$\dfrac{1}{7}$}{$\dfrac{1}{14}$}{Kitas atsakymas}}

%J29===============================[=============================.]===============================%J29
\def\JdnineEN{There are $n$ distinct lines on the plane, labeled $\ell_1, \dots, \ell_n$. The line $\ell_1$ intersects exactly $5$ other lines, the line $\ell_2$ intersects exactly $9$ other lines, and the line $\ell_3$ intersects exactly $11$ other lines. 
Which of the following is a smallest possible value of $n$?}
\def\JdnineLT{Plokštumoje pasirinktos $n$ skirtingų tiesių $t_1$, $t_2$, $\ldots$, $t_n$ ir pažymėtos visos jų sankirtos. Tiesėje $t_1$ yra lygiai $5$ sankirtos, tiesėje $t_2$ -- lygiai $9$, o tiesėje $t_3$ -- lygiai $11$. 
Kokios yra galimos skaičiaus $n$ reikšmės?}
\def\JdninePL{}
\def\JdnineRU{}

\def\rJdnine{E}
\def\aJdnine{\answer{Tik 14}{13 ir 14}{Tik 13}{12 ir 13}{Tik 12}}

%J30===============================[==============================]===============================%J30
\def\JdtenEN{Suppose $m$ and $n$ are integers with $0<m<n$. Let $P=(m, n)$, $Q=(n, m)$, and $O=(0,0)$. For how many pairs of $m$ and $n$ will the area of triangle $OPQ$ be equal to 2024?}
\def\JdtenLT{Koordinačių plokštumoje duoti taškai $P=(m, n)$, $Q=(n, m)$, ${O=(0,0)}$, kur skaičiai $m$ ir $n>m$ yra natūralieji (žr.~pav.).  Trikampio $OPQ$ plotas lygus $54$. Kiek skirtingų reikšmių gali įgyti suma $m+n$?}
\def\JdtenPL{}
\def\JdtenRU{}

\def\rJdten{A}
\def\aJdten{\answer{$2$}{$3$}{$4$}{$5$}{Kitas atsakymas}}

